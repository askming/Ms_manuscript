%%%%%%%%%%%%%%%%%%%%%%%%%%%%%%%%%%%%%%%%%%%%%%%%%%%%%%%%%%%%%%%%%%%%%%%%%%%%%%%%%%%%%%%%%%%%%%%%%%%%%%%%%%%%%%%%%%%%%%%%%%%%%%%%%%%%%%%%%%%%%%%%%%%%%%%%%%%
% This is just an example/guide for you to refer to when submitting manuscripts to Frontiers, it is not mandatory to use frontiers.cls nor frontiers.tex  %
% This will only generate the Manuscript, the final article will be typeset by Frontier after acceptance.                                                 %
%                                                                                                                                                         %
% When submitting your files, remember to upload this *tex file, the pdf generated with it, the *bib file (if bibliography is not within the *tex) and all the figures.
%%%%%%%%%%%%%%%%%%%%%%%%%%%%%%%%%%%%%%%%%%%%%%%%%%%%%%%%%%%%%%%%%%%%%%%%%%%%%%%%%%%%%%%%%%%%%%%%%%%%%%%%%%%%%%%%%%%%%%%%%%%%%%%%%%%%%%%%%%%%%%%%%%%%%%%%%%%

%%% Version 2.5 Generated 2014/09/25 %%%
%%% You will need to have the following packages installed: datetime, fmtcount, etoolbox, fcprefix, which are normally inlcuded in WinEdt. %%%
%%% In http://www.ctan.org/ you can find the packages and how to install them, if necessary. %%%

%\documentclass{frontiersENG} % for Engineering articles
%\documentclass{frontiersSCNS} % for Science articles
\documentclass{frontiersHLTH} % for Health articles
%\documentclass{frontiersFPHY} % for Physics articles

%\setcitestyle{square}
\usepackage{url,lineno}
\linenumbers

% Leave a blank line between paragraphs in stead of using \\

\copyrightyear{}
\pubyear{}

\def\journal{Health}%%% write here for which journal %%%
\def\DOI{}
\def\articleType{Research Article}
\def\keyFont{\fontsize{8}{11}\helveticabold }
\def\firstAuthorLast{Ming Yang {et~al.}} %use et al only if is more than 1 author
\def\Authors{Ming Yang\,$^{1}$, Dejian Lai\,$^{1,*}$ and Kim Waller\,$^2$}
% Affiliations should be keyed to the author's name with superscript numbers and be listed as follows: Laboratory, Institute, Department, Organization, City, State abbreviation (USA, Canada, Australia), and Country (without detailed address information such as city zip codes or street names).
% If one of the authors has a change of address, list the new address below the correspondence details using a superscript symbol and use the same symbol to indicate the author in the author list.
\def\Address{$^{1}$Department of Biostatistics, School of Public Health, the University of Texas Health Science Center at Houston, Houston, TX, USA \\
$^{2}$Division of Epidemiology, Human Genetics \& Environmental Sciences, School of Public Health, the University of Texas Health Science Center at Houston, Houston, TX, USA  }
% The Corresponding Author should be marked with an asterisk
% Provide the exact contact address (this time including street name and city zip code) and email of the corresponding author
\def\corrAuthor{Dejian Lai}
\def\corrAddress{Department of Biostatistics, School of Public Health, the University of Texas Health Science Center at Houston, 1200 Pressler, Houston, TX, 77030, USA}
\def\corrEmail{dejian.lai@uth.tmc.edu}

% \color{FrontiersColor} Is the color used in the Journal name, in the title, and the names of the sections.
\usepackage{multirow}
\graphicspath{{figure/}}

\begin{document}
\onecolumn
\firstpage{1}

\title[{\tiny A Study of Time Trend in Global Infant Mortality Rates}]{A Study of Time Trend in Global Infant Mortality Rates: Regression with Autocorrelated Data}
\author[\firstAuthorLast ]{\Authors}
\address{}
\correspondance{}
\extraAuth{}% If there are more than 1 corresponding author, comment this line and uncomment the next one.
%\extraAuth{corresponding Author2 \\ Laboratory X2, Institute X2, Department X2, Organization X2, Street X2, City X2 , State XX2 (only USA, Canada and Australia), Zip Code2, X2 Country X2, email2@uni2.edu}
\topic{}% If your article is part of a Research Topic, please indicate here which.

\maketitle

%%%%%%%%%%%%%%%%%%%%%%%%%%%%%%%%%%%%%%%%%%%%%%%%%%%%%%%%%%%%%%%%%%%%%%%%%%%%%%%%%%%%%%%%%%%%%%%%%%%%%%%%%%%%%%%%%%%%%%%%%%%%%%%%%%%%%%%%%%%%%%%%%%%%%%%%%%%%%%%%%%%%%%%%%%%%%%%%%%%%%%%%%%%%%%%%%%%%%%%%%%%%%%%%%%%%%%%%%%%%%%%%%%%%%%%
%%% The sections below are for reference only.
%%%
%%% For Original Research Articles, Clinical Trial Articles, and Technology Reports the section headings should be those appropriate for your field and the research itself. It is recommended to organize your manuscript in the
%%% following sections or their equivalents for your field:
%%% Abstract, Introduction, Material and Methods, Results, and Discussion.
%%% Please note that the Material and Methods section can be placed in any of the following ways: before Results, before Discussion or after Discussion.
%%%
%%%For information about Clinical Trial Registration, please go to http://www.frontiersin.org/about/AuthorGuidelines#ClinicalTrialRegistration
%%%
%%% For Clinical Case Studies the following sections are mandatory: Abstract, Introduction, Background, Discussion, and Concluding Remarks.
%%%
%%% For all other article types there are no mandatory sections.
%%%%%%%%%%%%%%%%%%%%%%%%%%%%%%%%%%%%%%%%%%%%%%%%%%%%%%%%%%%%%%%%%%%%%%%%%%%%%%%%%%%%%%%%%%%%%%%%%%%%%%%%%%%%%%%%%%%%%%%%%%%%%%%%%%%%%%%%%%%%%%%%%%%%%%%%%%%%%%%%%%%%%%%%%%%%%%%%%%%%%%%%%%%%%%%%%%%%%%%%%%%%%%%%%%%%%%%%%%%%%%%%%%%%%%%

\begin{abstract}

%%% Leave the Abstract empty if your article falls under any of the following categories: Editorial Book Review, Commentary, Field Grand Challenge, Opinion or specialty Grand Challenge.
\section{}
%As a primary goal, the abstract should render the general significance and conceptual advance of the work clearly accessible to a broad readership. References should not be cited in the abstract.
The infant mortality rate (IMR) is considered to be one of the most important indices of a country's well being. Countries around the world and international organizations such as the World Health Organization are dedicating their resources, knowledge and energy to reduce the infant mortality rates (IMR). The well-known Millennium Development Goal 4 (MDG 4), which is an example of such commitment, aims to achieve a two-thirds reduction of the under-five mortality rate between 1990 and 2015.

Many statistical tools are developed for data analysis by assuming independence of observations. However, IMR data collected over time forms a time series. The repeated observations of IMR time series are usually not statistically independent. So in modeling the trend of IMR, it is necessary to account for these autocorrelations. In this article we proposed to use the general linear models to take into account the autocorrelations to model worldwide IMR. We compared results from general linear model with correlation structure to that from ordinary least squares method to investigate how significantly the estimates change. Our analysis showed that results from these two methods were different for global data but not for specific countries except for two special cases and the discrepancy could be significantly different when considering the population size of the countries.

We modeled the trends of IMR from the 1950s to 2010s for selected countries. Our results quantified the trends of IMR over time and measured the difference across countries.

%Refer to \\ \url{http://www.frontiersin.org/}\texttt{\journal}\url{/authorguidelines} \\ or \textbf{Table\ref{Tab:01}} for abstract requirement and length according to article type.


\tiny
 \keyFont{ \section{Keywords:} Infant mortality rate,  Regression analysis,  Time series,  Autocorrelation} %All article types: you may provide up to 8 keywords; at least 5 are mandatory.
\end{abstract}

\section{Introduction}

% For Original Research Articles, Clinical Trial Articles, and Technology Reports the introduction should be succinct, with no subheadings.
%
% For Clinical Case Studies the Introduction should include symptoms at presentation, physical exams and lab results.
%
Infant mortality rate (IMR) is defined as the number of deaths within one year of age per 1,000 live births \cite{CDC}. The IMR is considered to be one of the important indicators of a population's health and well being along with life expectancy (LE) and under-five mortality rate (U5MR) \cite{Carl}. In the year of 2010, the IMR of the United States was 6.6 per 1000, compared with 66.9 per 1000 in Sudan \cite{CME}. This tenfold difference in IMR reveals a large gap in health conditions between these two nations. As IMRs from developed countries are generally lower than those in developing countries, the IMR is also a reflection of general socioeconomic conditions across different countries.

With improvement in socioeconomic conditions, improvement in health care for deliveries and newborns, and with recent efforts made by the World Health Organization (WHO), the IMR has been decreasing in most countries of the world \cite{UNICEF}. According to the Millennium Development Goal 4 (MDG 4), the global mortality rates for children who are younger than five years old are supposed to be reduced by two-thirds from 1990 to 2015 \cite{WHO}.

Although the IMR has declined to a relatively low level in some countries in recent decades, there are still many infants, especially those from the developing countries, who are suffering from many risk factors that lead to high IMR in those countries. Our study of time trend of IMR over the past a few decades will provide the health organizations and the general public a clear insight into this global health issue. In our article, we have investigated recent achievements regarding reductions in the IMR, globally with a special attention on specific nations or regions. We modeled the current trend of IMRs and measured gaps from the goal as set in the MDG4 \cite{Julie}, etc. The results from this study will help us to better understand the trend of IMR to formulate strategies for improvement in the IMRs and to decide where to focus our attention and efforts and how to allocate the health resources around the world more effectively\cite{SinghGK}.

%\begin{methods}
\section{Material \& Methods}

\label{sec:method}

\subsection{Study Design and Study Population}%%%%%%%
\label{sec:design}
In this study we used countries as the study units \cite{Carl}. For each of the countries, annual IMR data were collected and the earliest time point in records starts from 1932. In our regression analysis model, the response variable is the IMR and the independent variable is the time (year). 

\subsection{Data}%%%%%%%
\label{sec:data}
Our data set includes annual infant mortality rate for 195 countries/regions around the world between the times of May 1932 to May 2010. The data set is mainly collected from the Child Mortality Estimates (CME)\footnote{CME Info is a database containing the latest child mortality estimates based on the research of the UN Inter-agency Group for Child Mortality Estimation.}  website: http://www.childmortality.org \cite{CME}, compiled with those from the World Bank website \cite{WorldBank}. A closer examination of the data set showed that for most of the countries/regions, the data were mostly missing before 1950. To deal with this issue, we set the time interval starting from 1950 for our study. Countries with sparse data were not included in the study. Also we took consideration to cover countries from different continents and different level of socioeconomic status in order to investigate the influence of those factors on the trend of the IMR. To make it representative and comprehensive, we selected the countries and regions that are listed in Table 1.

\begin{table}[!t]
\textbf{\refstepcounter{table}\label{Tab:01} Table \arabic{table}.}{ Selected countries for the study }

\processtable{ }
{\begin{tabular}{l l l}\toprule
{Continent}& {Countries Selected} & {Development Status}\\ \midrule
\multirow{2}{*}{Asia} & Japan, Singapore& developed (countries) \\
 & China, India, Thailand & developing (countries)\\[2mm]

\multirow{2}{*}{Europe} & UK, Italy, Sweden & developed \\
 & Russia, Poland & developing \\[2mm]

Africa & Algeria, Sudan, Zambia,Morocco, Libya & developing\\[2mm]
\multirow{2}{*}{America} & US&developed \\
 & Brazil, Mexico & developing \\[2mm]
 Oceania & Australia & developed\\\botrule
\end{tabular}}{}
\end{table}

\subsection{Statistical Method}%%%%%%%
\label{sec:stat method}

\subsubsection{General linear models and generalized lest squares}%%%%%%%
\label{sec:GLS}

For the linear regression model, i.e. $y_{n\times1}=\bf{X}_{n\times p} \beta_{p\times1}+\varepsilon_{n\times1}$ \footnote{$n$ indicates the number of observations, $p$ indicates the number of covariates}, we assume the error term $\varepsilon\sim N_{n}(0,\sigma^2\bf{I_{n}})$  ,which leads to the \emph{ordinary-least-squares(OLS)} estimator of $\beta$ \cite{Fox}:

\begin{equation} 
\boldsymbol{\hat{\beta}_{OLS}=(X'X)^{-1}X'y } %Eqn.(1)
\end{equation}

\noindent \textmd{with variance-covariance matrix:}
\begin{equation}
{Var\boldsymbol{(\hat{\beta}}_{OLS})=\sigma^2(X'X)^{-1}} %Eqn.(2)
\end{equation}

However, in our study this ideal assumption of error terms cannot always be satisfied--there may be some correlations among the error terms, which can be modeled with a more general assumption of $\varepsilon\sim N_{n}(0,\bf{\Sigma)}$ , where the error-covariance matrix $\bf{\Sigma}$ is symmetric and positive-definite. Different diagonal entries in $\bf{\Sigma}$ represent non-constant error variances and nonzero off-diagonal entries correspond to correlated errors. Based on this more general assumption, our estimator of $\beta$ becomes the \emph{generalized-least-squares (GLS)} estimator if the variance-covariance matrix is known:

\begin{equation}
\boldsymbol{\hat{\beta}_{GLS}=(X'\Sigma^{-1}X)^{-1}X'\Sigma^{-1}y}%Equation (3)
\end{equation}

\noindent \textmd{with variance-covariance matrix:}
\begin{equation}
\bf{Var\boldsymbol{(\hat{\beta}}_{GLS})=(X'\Sigma^{-1}X)^{-1}} %Eqn.(4)
\end{equation}


\noindent \textmd{Time-series data is a typical example in which the error terms from a regression model are correlated, since the data are obtained from multiply measurements on the same subject over time \cite{Chris}. In our study, the observations of infant mortality rates were collected annually from each of the countries, which means that the data are equally spaced with intervals of one-year period. This type of data belongs to discrete time series case \cite{Chris}. The IMR data measured sequentially over time in a given country are very likely to be correlated due to the similarity of the socioeconomic condition, health care condition, etc. within that country. Ignoring correlations in the model building steps may lead to inaccurate or even misleading results.}\cite{Diggle}

\subsubsection{Time-series regression models}%%%%%%%
\label{sec:timeseries}

In this study, we assume that the  regression errors is \emph{stationary} \cite{Peter1} \cite{Peter2}, that is, the covariance of two error terms depends only upon their separation k in time and is independent of time t, i.e. $cov(\varepsilon_{t},\varepsilon_{t+k})=cov(\varepsilon_{t}, \varepsilon_{t-k})=\sigma^2\rho_{k}$, where $\rho_{k}<1$ is the error autocorrelation at lag k \cite{Peter2}. (With underlying assumptions $E(\varepsilon)=0$ and  $var(\varepsilon_{t})=\sigma^2)$ \newline
Thus, the error-covariance matrix can be written as following:

\begin{equation}
\bf{\Sigma}=\sigma^2
\begin{bmatrix}
1 & \rho_{1} & \rho_{2} & \rho_{3} & \ldots &\rho_{n-1}\\
\rho_{1} & 1 & \rho_{1} & \rho_{2} & \ldots &\rho_{n-2}\\
\rho_{2} & \rho_{1} & 1 & \rho_{1} & \ldots &\rho_{n-3}\\
\vdots& \vdots & \vdots & \vdots & \ddots & \vdots\\
\rho_{n-1} &\rho_{n-2} & \rho_{n-3} & \rho_{n-4} & \ldots &1
\end{bmatrix}
=\sigma^2\bf{P}
\end{equation}

Since the true value of $\sigma^2$  and $\rho$'s  are unknown, we cannot directly apply above result to finding the GLS estimator of $\beta$ in this time-series regression setting. In addition, the large number, (n-1), of $\rho$'s makes it impossible to estimate them without specifying any structure for the autocorrelated errors.\\

\noindent \emph{The first-order autoregression process, AR(1):}
\begin{equation}
\varepsilon_{t}=\phi\varepsilon_{t-1}+\upsilon_{t} ,
\end{equation}

\noindent \textmd{where $\varepsilon_{t}$ and $\varepsilon_{t-1}$ are the error terms at time t and t-1 respectively; $\upsilon_{t}$ is the \emph{Gaussian white noise}, $N(0,\sigma^2_{\upsilon})$. In this model, it can be shown that $\rho_{1}=\phi, \rho_{k}=\phi^k$, and $\sigma^2=\sigma^2_{\upsilon}/{(1-\phi^2)}$ \cite{Peter2}. Because $|\phi|<1$ is assumed for stationary autoregressive processes, the error autocorrelation $\rho_{k}$ approaches to 0 exponentially as s increases.} The variance-covariance matrix under AR(1) model can be expressed as

\begin{equation}
\bf{\Sigma}=\sigma^2
\begin{bmatrix}
1 & \phi & \phi^2 & \phi^3 & \ldots &\phi^{n-1}\\
\phi & 1 & \phi& \phi^2 & \ldots &\phi^{n-2}\\
\phi^2& \phi & 1 & \phi & \ldots &\phi^{n-3}\\
\vdots& \vdots & \vdots & \vdots & \ddots & \vdots\\
\phi^{n-1} &\phi^{n-2} & \phi^{n-3} & \phi^{n-4} & \ldots &1
\end{bmatrix}
=\sigma^2\bf{P}
\end{equation}

In our modelling, we first fitted the data using linear regression model by assuming ${\bf P=I}$; after that and before deciding whether it's appropriate to use GLS method to model the data, we tested the correlation among the observations with two methods: one is the Durbin-Watson test and the other is plotting the correlogram.\cite{correlogram} When the existence of autocorrelation was detected, we built the model for the IMR data based on time-series and compared the results with those from OLS model.



%
%
%Text Text Text Text Text Text  Text Text Text Text Text Text Text Text Text  Text Text Text Text Text Text Text Text Text Text  Text Text Text Text Text Text  Text Text.  \cite{Neuro2013} might want to know about  text text text text Text Text Text Text  Text Text Text Text Text Text  Text Text. \citep{Gene2012} might want to know about  text text text text
%Text Text Text Text Text Text  Text Text Text Text Text Text Text Text Text  Text Text Text Text Text Text Text Text Text Text  Text Text Text Text Text Text  Text Text.  \cite{Neurobot2013} might want to know about  text text text text
%
%\begin{table}[!t]
%\textbf{\refstepcounter{table}\label{Tab:01} Table \arabic{table}.}{ Maximum size of the Manuscript }
%
%\processtable{ }
%{\begin{tabular}{lllll}\toprule
% & Abstract max. legth (incl. spaces) & Figures or tables & Manuscript max. length & Final PDF length\\\midrule
%Clinical Case Study & & & &\\
%Clinical Trial & & & &\\
%Hypothesis and Theory & & & &\\
%Methods & 2000 characters  & 15 & 12000 words & 12 pages\\
%Original Research & & & &\\
%Review & & & &\\
%Technology Report & & & &\\
%Focused Review & 2000 characters & 5 & 5000 words & 5 pages\\
%CPC &  1250 characters& 6 & 2500 words & 4 pages\\
%Perspective & 1250 characters & 2 & 3000 words & 3 pages\\
%Mini Review & & & &\\
%Classification & 1250 characters & 10 & 2000 words & 12 pages\\
%Editorial & none & none & 1000 words & 1 page \\
%Book review & & & &\\
%Frontiers Commentary & none & 1 & 1000 words & 1 page\\
%General Commentary & & & &\\
%Field Grand Challenge & & & &\\
%Opinion & none & 1 & 2000 words & 2 pages\\
%Specialty Grand Challenge& & & &\\\botrule
%\end{tabular}}{}
%\end{table}
%
%Please note that very large tables (covering several pages) cannot be included in the final PDF for reasons of space. These tables will be published as supplementary material on the online article abstract page at the time of acceptance. The author will notified during the typesetting of the final article if this is the case. A link in the final PDF will direct to the online material.
%
%\subsection{Original Research Articles, Clinical Trial Articles, and Technology Reports}
%
%For Original Research Articles, Clinical Trial Articles, and Technology Reports the section headings should be those appropriate for your field and the research itself. It is recommended to organize your manuscript in the following sections or their equivalents for your field:
%
%\begin{itemize}
%%for bulleted list, use itemize
%\item Introduction: Succinct, with no subheadings.
%\item Materials and Methods: This section may be divided by subheadings. This section should contain sufficient detail so that when read in conjunction with cited references, all procedures can be repeated.
%\item Results: This section may be divided by subheadings. Footnotes should not be used and have to be transferred into the main text.
%\item Discussion: This section may be divided by subheadings. Discussions should cover the key findings of the study: discuss any prior art related to the subject so to place the novelty of the discovery in the appropriate context; discuss the potential short-comings and limitations on their interpretations; discuss their integration into the current understanding of the problem and how this advances the current views; speculate on the future direction of the research and freely postulate theories that could be tested in the future.
%\end{itemize}
%
%Please note that the Material and Methods section can be placed in any of the following ways: before Results, before Discussion or after Discussion.
%
%\subsection{Clinical Case Studies}
%
%For Clinical Case Studies the following sections are mandatory:
%
%\begin{itemize}
%%for bulleted list, use itemize
%\item Introduction: Include symptoms at presentation, physical exams and lab results.
%\item Background: This section may be divided by subheadings. Include history and review of similar cases.
%\item Results: This section may be divided by subheadings. Include diagnosis and treatment.
%\item Concluding Remarks
%\end{itemize}
%
%%\end{methods}



\section{Results}
\label{sec:resutls}



The trends in IMR between 1950 and 2010 are depicted in \textbf{Figure \ref{fig:01}}, in which the three decreasing curves represent the time trends in global IMR, in developed countries and developing countries\footnote{Developed countries/regions comprise Europe, Northern America, Australia/New Zealand and Japan; developing countries/regions comprise all regions of Africa, Asia (excluding Japan), Latin America and the Caribbean plus Melanesia, Micronesia and Polynesia.} respectively based on the data from World Bank \cite{WorldBank}. \textbf{Figure \ref{fig:01}} shows that the IMR in developed countries is much lower compared with that in developing countries and the overall global IMR, which is calculated by combining all the countries in previous two categories, lies between them. We also plotted the time trend in IMR for each of the selected countries (see \textbf{Figure \ref{fig:03}} to \textbf{Figure \ref{fig:06}}) with the IMR values from 1950 to 2010.\\

Generally, it can be observed from these plots that although the IMR from developing countries was much higher, it decreased more rapidly than the IMR in developed countries. The difference of IMR between them is much smaller in 2010 compared with the difference of IMR in 1950 as shown in the figures. The absolute value of the slope of the regression lines modeling the time trend in IMR of developing countries is larger than that in developed countries.\\

To investigate the relationship between IMR and time quantitatively, we implemented the method of regression analysis and built models of IMR in terms of time by estimating the regression coefficient of the model. We first tried simple OLS method without taking into account the autocorrelation to model the IMR time trend, then we built new models that take into account autocorrelation. The results from these two methods were compared. We took the United Stated as an example to conduct the above analyses. The results of the other selected countries were obtained similarly.

\subsection{Fitted Models with OLS Method}
Without considering the autocorrelation among the observations, first we simply tried to fit the regression model for the IMR data by OLS method. Table 2 (upper) shows the estimations of coefficients using the OLS regression of United States (1950--2010) and \textbf{Figure \ref{fig:02}} (left) shows the IMR trend in time of United States between the years of 1950 to 2010, along with the regression line fitted to the data. The function shows the relationship between IMR and time:

\[IMR=-0.456*year+917.8335,\]

which means that the IMR of United States decreased 0.456 per 1000 per year during this time interval.

\begin{table}[!t]
\textbf{\refstepcounter{table}\label{Tab:02} Table \arabic{table}.}{ Regression coefficients of IMR v.s. time from OLS (upper) \& GLS (lower) models, United States (1950$\sim$2010) }

\processtable{ }
%\caption{Regression coefficients of IMR v.s. time from OLS (upper) \& GLS (lower) models, United States (1950$\sim$2010)}
{
\begin{tabular}{rrrrr}\toprule
 & Estimate & Std. Error & t value & Pr($>$$|$t$|$) \\ \midrule
(Intercept) & 917.8335 & 29.8878 & 30.71 & 0.0000 \\ 
  Year & -0.4556 & 0.0151 & -30.19 & 0.0000 \\ \botrule
\end{tabular}}{}

%\caption{Results from GLS regression, United States (1950$\sim$2010)}
\processtable{ }{
\begin{tabular}{llll}\toprule
 & $\rho$ & Intercept & Slope \\ \midrule
 est. & 0.9926 & 864.2053  & -0.4271 \\ 
  s.e.& 0.0071 & 59.1964 & 0.0299 \\\botrule
\end{tabular}}{}
\end{table}




\subsection{Fitted Models with GLS Method}
The \verb'R' function \verb'arima' was used to build the time-series regression model. For the data from United States we chose the order (1,0,0), which equals to the first order autoregressive model� i.e. AR(1).
Table 2 (lower) shows the output, and the resulting model is:

\[IMR=-0.427*year+864.2053,\]

which means that the IMR of United States decreased 0.427 per 1000 per year during the time period between 1950 and 2010, and with $\rho=0.9926$, $\sigma^2=0.06588$. The regression line is shown in \textbf{Figure \ref{fig:02}} (right) along with the IMR trend curve similar as in \textbf{Figure \ref{fig:02}} (left). 


\subsection{Comparison of time trends of the IMR from the selected countries}

\textbf{Figure \ref{fig:03}} to \textbf{Figure \ref{fig:06}} show the country-wise time trends of the IMR by continent.
To compare the equality of the decreasing rate (the slope) of the IMR between developed and developing countries, we applied the Wilcoxon rank sum test. The Wilcoxon rank sum test gave us p-values of 0.0008 in testing the equality of slopes in models for developed countries and developing countries from OLS method and 0.0003 from the GLS method respectively. \\

To compare the slopes for two specific countries, we implemented two-sample {\em t-test} (based on the relatively large sample size, which is around 60 for the selected countries). And the results are listed in \textbf{Table \ref{Tab:03}}. \footnote{All slopes are negative. Positive difference indicates that first country has smaller slope, in absolute value, than the second, indicating smaller decreasing rate of IMR, vice versa.}

\begin{table}[!t]
\textbf{\refstepcounter{table}\label{Tab:03} Table \arabic{table}.}{ Comparison of the slopes (from GLS) for different countries }

\processtable{ }
%\caption{Comparison of the slopes (from GLS) for different countries }
{\begin{tabular}{lccccccc}\toprule
$Comparison$ & $Diff.$ $in$ $slopes$ & $Std$ $of$ $diff.$ & $z-value$ & $p-value$&$lower$ & $upper$\\\midrule 

US-UK & 0.0113 & 0.042 & 0.266 & 0.790 & -0.072 & 0.095\\ 
Australia-Sweden & -0.036 & 0.031 & -1.173 & 0.241 & -0.096 & 0.024\\ 
Japan-Singapore & 0.0822 & 0.164 & 0.502 & 0.616 & -0.239 & 0.403\\ 
China-India & 0.6751 & 0.163 & 4.145 &{\bf$ <$0.00001} & 0.356 & 0.994\\ 
Sudan-Zambia & -0.1078 & 0.184 & -0.586 & 0.558 & -0.468 & 0.252\\ 
US-China & 1.2523 & 0.135 & 9.266 & {\bf$<$0.00001} & 0.987 & 1.517\\\botrule
\end{tabular}}{}
\end{table}


\subsection{Comparison of the results from two model building methods}
In \textbf{Table \ref{Tab:04}} we listed the slopes and corresponding standard errors for all the countries from both models. As mentioned previously, regardless of the close value of the coefficients, they still can be statistically different. To check if the regression coefficients from two regression methods for each specific country are statistically different, we assumed that the two estimates were independent and conducted a series of {\em z-tests}. The test results are listed in \textbf{Table \ref{Tab:05}}. To check the overall similarity of the two model-building methods, we treated the 19 slope estimates from OLS and GLS methods as two groups respectively and conducted the two-sample paired {\em t-test}. 


\begin{table}[!t]
\textbf{\refstepcounter{table}\label{Tab:04} Table \arabic{table}.}{ Regression coefficients from two model-building methods }

\processtable{ }
%\caption{Regression coefficients from two model-building methods}
{\begin{tabular}{lccccccc}\toprule

$country$& $development\_status$ & $ols\_slope$ & $s.e.$ & &$gls\_slope$ & $s.e.$ &$\rho$ \\ \midrule
Japan & developed &  -0.7292 & 0.05377 && -0.8995 & 0.1145 & 0.9126  \\ 
Singapore & developed &  -0.8231 & 0.04683 && -0.9817 & 0.1172 & 0.8773 \\ 
Thailand &  developing & -2.06 & 0.06825 && -2.066 & 0.1504 & 0.9035 \\ 
India &  developing & -2.275 & 0.04358 & &-2.3545 & 0.0957 & 0.8918 \\ 
China & developing & -1.454 & 0.05659 && -1.6794 & 0.1318 & 0.8563 \\ 
Algeria &  developing & -2.635 & 0.06008 && -2.5626 & 0.1308 & 0.9185 \\ 
Sudan &  developing & -1.15 & 0.05225 && -1.3219 & 0.1156 & 0.9068 \\ 
Zambia &  developing & -0.8197 & 0.05924 && -1.2141 & 0.1429 & 0.9383 \\ 
Libya &  developing & -3.046 & 0.1895 && -3.7296 & 0.4409 & 0.8859 \\ 
Morocco & developing & -2.267 & 0.02545 && -2.1171 & 0.0602 & 0.9256 \\ 
UK &  developed & -0.4382 & 0.0122 && -0.4384 & 0.0302 & 0.8991 \\ 
Sweden &  developed & -0.3094 & 0.009713 && -0.3149 & 0.0214 & 0.9253 \\ 
Italy &  developed & -0.9982 & 0.04317 && -1.0559 & 0.093 & 0.9125 \\ 
Poland &  developing & -1.008 & 0.05758 && -1.263 & 0.15 & 0.8592 \\ 
Russia &  developing & -0.5489 & 0.01509 && -0.5686 & 0.0289 & 0.9181 \\ 
US & developed & -0.4556 & 0.01509 && -0.4271 & 0.0299 & 0.9367 \\ 
Brazil &  developing & -2.427 & 0.04272 && -2.4057 & 0.093 & 0.9132 \\ 
Mexico &  developing & -1.741 & 0.02986 && -1.6635 & 0.0587 & 0.9235 \\ 
Australia &  developed & -0.3784 & 0.008724 && -0.3509 & 0.022 & 0.9310 \\\botrule
\end{tabular}}{}
\end{table}%


\begin{table}[!t]
\textbf{\refstepcounter{table}\label{Tab:05} Table \arabic{table}.}{ Statistical test of the equivalence of the slope estimates from OLS and GLS methods }

\processtable{ }
%\caption{ Statistical test of the equivalence of the slope estimates from OLS and GLS methods }
{\begin{tabular}{lrcccccc}\toprule
$Country$ & $Diff.$ & $Std$ $of$ $Diff.$ & $z-value$ & $p-value$&$lower$ & $upper$\\\midrule
Japan & 0.1703 & 0.126 & 1.346 & 0.177 & -0.078 & 0.418\\ 
Singapore & 0.1586 & 0.126 & 1.257 & 0.208 & -0.089 & 0.406\\ 
Thailand & 0.006 & 0.165 & 0.036 & 0.971 & -0.318 & 0.330\\ 
India & 0.0795 & 0.105 & 0.756 & 0.450 & -0.127 & 0.286\\ 
China & 0.2254 & 0.143 & 1.571 & 0.116 & -0.056 & 0.507\\ 
Algeria & -0.0724 & 0.144 & -0.503 & 0.615 & -0.355 & 0.210\\ 
Sudan & 0.1719 & 0.127 & 1.355 & 0.175 & -0.077 & 0.421\\ 
Zambia & 0.3944 & 0.155 & 2.550 & {\bf0.011} & 0.091 & 0.698\\ 
Libya & 0.6836 & 0.480 & 1.424 & 0.154 & -0.257 & 1.624\\ 
Morocco & -0.1499 & 0.065 & -2.294 & {\bf0.022} & -0.278 & -0.022\\ 
UK & 0.0002 & 0.033 & 0.006 & 0.995 & -0.064 & 0.064\\ 
Sweden & 0.0055 & 0.024 & 0.234 & 0.815 & -0.041 & 0.052\\ 
Italy & 0.0577 & 0.103 & 0.563 & 0.573 & -0.143 & 0.259\\ 
Poland & 0.255 & 0.161 & 1.587 & 0.113 & -0.060 & 0.570\\ 
Russia & 0.0197 & 0.033 & 0.604 & 0.546 & -0.044 & 0.084\\ 
US & -0.0285 & 0.033 & -0.851 & 0.395 & -0.094 & 0.037\\ 
Brazil & -0.0213 & 0.102 & -0.208 & 0.835 & -0.222 & 0.179\\ 
Mexico & -0.0775 & 0.066 & -1.177 & 0.239 & -0.207 & 0.052\\ 
Australia & -0.0275 & 0.024 & -1.162 & 0.245 & -0.074 & 0.019\\\botrule
\end{tabular}}{}
\end{table}%







%
%Frontiers requires figures to be submitted individually, in the same order as they are referred to in the manuscript. Figures will then be automatically embedded at the bottom of the submitted manuscript. Kindly ensure that each table and figure is mentioned in the text and in numerical order. Permission must be obtained for use of copyrighted material from other sources (including the web). Please note that it is compulsory to follow figure instructions. Figures which are not according to the guidelines will cause substantial delay during the production process.
%
%\begin{table}[!t]
%\textbf{\refstepcounter{table}\label{Tab:02} Table \arabic{table}.}{ Resolution Requirements for the figures}
%
%\processtable{}
%{\begin{tabular}{lllll}\toprule
%Image Type & Description & Format & Color Mode & Resolution\\\midrule
%Line Art & An image composed of lines and text,  & TIFF, JPEG & RGB, Bitmap & 900 - 1200 dpi\\
%           & which does not contain tonal or shaded areas.& & &\\
%           Halftone & A continuous tone photograph, which contains no text. & TIFF, EPS, JPEG & RGB, Grayscale & 300 dpi\\
%Combination & Image contains halftone + text or line art elements. & TIFF, JPEG & RGB,Grayscale & 600 - 900 dpi\\\botrule
%\end{tabular}}{}
%\end{table}
%
%\begin{equation}
%\sum x+ y =Z\label{eq:01}
%\end{equation}
%
%\textbf{Table\ref{Tab:02}} shows the resolution requirements for the figures. The figures must be legible:
%\begin{enumerate}
%\item The smallest visible text is no less than 8 points in height, when viewed at actual size.
%\item Solid lines are not broken up.
%\item Image areas are not pixelated or stair stepped.
%\item Text is legible and of high quality.
%\item Any lines in the graphic are no smaller than 2 points width.
%\item The actual size of the figure must be of at least 8.5 cm.
%\end{enumerate}

\section{Discussion}

Time trend of the IMR were studied for selected countries in this study, in which the graphical representation showed how the IMR has declined during the past six decades. After investigating the autocorrelations among the observations using Durbin-Watson test as well as the autocorrelation function (acf) and partial-autocorrelation function (p-acf), both produced significant results, i.e. the IMR data are statistically correlated (not shown in this paper). We proceeded to build up the time-series regression models for the IMR in term of time based on the results from autocorrelation tests and then compared them with those results from OLS methods.

\subsection{Time trends of the IMR for different regions}
Among all the selected countries, the situation of IMR was the worst in Africa and the best in Europe and Oceania. Although they all have achieved a huge decline in IMR for the past six decades, the general IMR in Africa is still around the level that the European countries had in 1950s (around 30 per 1000). One special case is in Zambia, instead of a decrease, the IMR increased during the 1980s to 1990s. In Asia and America the case was similar. There was still a gap of IMR between developing and developed countries, however the developing countries are catching up.\\

The baseline of IMR was much higher in developing countries compared with developed countries; however the difference became smaller during the past six decades since the IMR decreased faster in those developing countries. Wilcoxon rank sum tests showed that the decreasing rates between developing countries and developed countries were significantly higher in developing countries either based on OLS or GLS method (with {\em p-values} 0.00081 and 0.0003 respectively). \\

From \textbf{Table \ref{Tab:03}}, for example, we can see that the slopes for China and India are significantly different (with difference equals to 0.6751 and {\em p-value} $<$0.0001), which indicates China has smaller IMR decreasing rate than India (in absolute value of the slopes). The same conclusion can be drawn form the comparison between US and China (US has smaller IMR decreasing rate than China). Comparisons can also be done for other countries using similar method. Although high achievement had been made in reducing the IMR in developing countries for the past few decades, attentions, efforts, and resources are still desirable and necessary in the future. For the countries that already have lower IMR, more effective methods need to be considered and applied to lower the IMR further.

\subsection{The difference in two model-building methods}
Two regression lines were presented in \textbf{Figure \ref{fig:03}} to \textbf{Figure \ref{fig:06}}, graphically showing the comparison between these two model-building methods (solid line stands for OLS regression and dashed line stands for GLS regression). Neither from the numeric values in Table 4 nor from the graphs the results from OLS method are greatly different from those from the GLS method at the first glance -- they have similar absolute values or the two regression lines lie close to each other. Country-wise statistical tests for comparison of the difference in the estimates of slopes also turned out to be insignificant expect for two countries as we can see from \textbf{Table \ref{Tab:05}}: Zambia and Morocco had significant results, which means the regression slopes from OLS and GLS methods are significantly different with each other. For all other countries we failed to reject the null hypothesis and both OLS and GLS methods produced similar estimates of slopes.\\

However the overall two-sample paired {\em t-test} showed that the regression coefficients from the OLS and GLS methods are statistically significantly different from each other with {\em p-value} turned out to be 0.0435, which was slightly less than 0.05, indicated we could reject the null hypothesis, and the slops from OLS and GLS methods were statistically different with a significance level of 5$\%$. In addition, when we take account of the whole population size, this ``small" difference could result in huge discrepancy: for example, the population of US was around 308,745,538  in 2010. If we apply above two methods separately the difference in IMR for the year of 2010 can be calculated as

\[|(-0.427 *2010+ 864.2053)-(-0.456 *2010 + 917.8335)|=4.6618 / 1000\]

If we multiply this value with the population of US, the difference is 1,439,310 deaths of infant, which is not minor. Same for other nations, when we take into account the population size, the OLS and GLS methods actually lead to different results that cannot be ignored. As we have shown that the IMR data are highly correlated, it is highly necessary to choose GLS method instead of simply apply OLS regression method.

%Text Text Text Text Text Text  Text Text Text Text Text Text Text Text Text  Text Text Text Text Text Text Text Text Text Text.
%Additional Requirements:


%\subsection{Corrections}
%
%If you need to communicate important changes to a published article please submit a General Commentary. Submit the article with the title “Corrigendum: Original Title of Article”.
%
%\subsection{Commentaries on Articles}
%
%At the beginning of your manuscript provide the citation of the article commented on.
%
%\subsection{Focused Reviews}
%
%For Tier 2 invited Focused Reviews the sections Introduction, Material and Methods, Results, and Discussion are recommended. In addition the authors must submit a short biography of the corresponding author(s). This short biography has a maximum of 600 characters, including spaces.
%
%A picture (5 x 5 cm, in *.tif or *.jpg, min 300 dpi) must be submitted along with the biography in the manuscript and separately during figure upload.
%Focused Reviews highlight and explain key concepts of your work. Please highlight a minimum of four and a maximum of ten key concepts in bold in your manuscript and provide the definitions/explanations at the end of your manuscript under “Key Concepts”. Each definition has a maximum of 400 characters, including spaces.
%
%\subsection{Human Search and Animal Research}
%
%All experiments on live vertebrates or higher invertebrates must be performed in accordance with relevant institutional and national guidelines and regulations. In the manuscript, authors must identify the committee approving the experiments and must confirm that all experiments conform to the relevant regulatory standards. For manuscripts reporting experiments on human subjects, authors must identify the committee approving the experiments and must also include a statement confirming that informed consent was obtained from all subjects. In Original Research Articles and Clinical Trial Articles these statements should appear in the Materials and Methods section.
%
%\subsection{Clinical Trial Registration}
%
%Clinical trials should be registered in a public trials registry in order to become the object of a publication at Frontiers. Trials must be registered at or before the start of patient enrollment. A clinical trial is defined as"any research study that prospectively assigns human participants or groups of humans to one or more health-related interventions to evaluate the effects on health outcomes."(\url{www.who.int/ictrp/en}). A list of acceptable registries can be found at \url{www.who.int/ictrp/en and www.icmje.org}.
%
%\subsection{Inclusion of Proteomics Data}
%
%Authors should provide relevant information relating to how the peptide/protein matches were undertaken, including methods used to process and analyze data, false discovery rates (FDR) for large-scale studies and threshold or cut-off rates for peptide and protein matches. Further information could include software used, mass spectrometer type, sequence database and version, number of sequences in database, processing methods, mass tolerances used for matching, variable/fixed modifications, allowable missed cleavages, etc.
%
%Authors should provide as supplementary material information used to identify proteins and/or peptides. This should include information such as accession numbers, observed mass (m/z), charge, delta mass, matched mass, peptide/protein scores, peptide modification, miscleavages, peptide sequence, match rank, matched species (for cross species matching), number of peptide matches, ambiguous protein/peptide matches should be indicated, etc.
%For quantitative proteomics analyses authors should provide information to justify the statistical significance including biological replicates, statistical methods, estimates of uncertainty and the methods used for calculating error.
%
%For peptide matches with biologically relevant post-translational modifications (PTM) and for any protein match that has occurred using a single mass spectrum, authors should include this information as raw data, annotated spectra or submit data to an online repository (recommended option).
%Authors are encouraged to submit raw or matched data and 2-DE images to public proteomics repositories. Submission codes and/or links to data should be provided within the manuscript.
%
%\subsection{Data Sharing}
%
%Frontiers supports the policy of data sharing, and authors are advised to make freely available any materials and information described in their article, and any data relevant to the article (while not compromising confidentiality in the context of human-subject research) that may be reasonably requested by others for the purpose of academic and non-commercial research. In regards to deposition of data and data sharing through databases, Frontiers urges authors to comply with the current best practices within their discipline.

\section*{Disclosure/Conflict-of-Interest Statement}
%Frontiers follows the recommendations by the International Committee of Medical Journal Editors (http://www.icmje.org/ethical_4conflicts.html) which require that all financial, commercial or other relationships that might be perceived by the academic community as representing a potential conflict of interest must be disclosed. If no such relationship exists, authors will be asked to declare that the research was conducted in the absence of any commercial or financial relationships that could be construed as a potential conflict of interest. When disclosing the potential conflict of interest, the authors need to address the following points:
%•	Did you or your institution at any time receive payment or services from a third party for any aspect of the submitted work?
%•	Please declare financial relationships with entities that could be perceived to influence, or that give the appearance of potentially influencing, what you wrote in the submitted work.
%•	Please declare patents and copyrights, whether pending, issued, licensed and/or receiving royalties relevant to the work.
%•	Please state other relationships or activities that readers could perceive to have influenced, or that give the appearance of potentially influencing, what you wrote in the submitted work.

The authors declare that the research was conducted in the absence of any commercial or financial relationships that could be construed as a potential conflict of interest.

\section*{Author Contributions}
%When determining authorship the following criteria should be observed:
%•	Substantial contributions to the conception or design of the work; or the acquisition, analysis, or interpretation of data for the work; AND
%•	Drafting the work or revising it critically for important intellectual content; AND
%•	Final approval of the version to be published ; AND
%•	Agreement to be accountable for all aspects of the work in ensuring that questions related to the accuracy or integrity of any part of the work are appropriately investigated and resolved.
%Contributors who meet fewer than all 4 of the above criteria for authorship should not be listed as authors, but they should be acknowledged. (http://www.icmje.org/roles_a.html)


This is originally MY's thesis work for his master's of science (MS) degree at the University of Texas School of Public Health. Dr. DL is the chair of the dissertation committee and Dr. KW is another committee member from the epidemiology division. 

\section*{Acknowledgement}
The author, Ming Yang, would like to thank Dr. Dejian Lai for his patient guidance and continuous support during this research project and Dr. Kim Waller for those precious time and advice. The author also would like to express his gratitude to his friends in UTSPH for their accompany and encouragement during his pursuit of the MS degree. 

\paragraph{Funding\textcolon} This study is not funded.

\section*{Supplemental Data}
See Reference items 4 and 7.

%\bibliographystyle{frontiersinSCNS&ENG} % for Science and Engineering articles
%\bibliographystyle{frontiersinHLTH&FPHY} % for Health and Physics articles
%\bibliography{test}

%%% Upload the *bib file along with the *tex file and PDF on submission if the bibliography is not in the main *tex file


\begin{thebibliography}{}
%
% and use \bibitem to create references. Consult the Instructions
% for authors for reference list style.
%
\bibitem{CDC} CDC. Eliminate the Disparities in Infant Mortality. \emph{Office of Minority Health \& Health Disparities (OMHD).}
\bibitem{Carl} Carl Otto Schell, Marie Reilly, Hans Rosling, Stefan Peterson and Anna Mia Ekstr. Socioeconomic determinants of infant mortality: A worldwide study of 152 low-, middle-, and high-income countries. \emph{Scand J Public Health} 2007; 35: 288--297.
\bibitem{Singh}Singh GK, Yu SM. Infant mortality in the United States: Trends, differentials, and projections, 1950 through 2010. \emph{Am J Public Health.} 1995; 85(7): 957--964.
\bibitem{CME}Child Mortality Estimates (CME) Info. \emph{Child Mortality Report 2011}. (accessed May 2012) {http://www.childmortality.org}
\bibitem{UNICEF} UNICEF. Statistics by area/Child survival and Health. \emph{Under-five Mortality} (accessed May 2012). \verb'http://www.childinfo.org'
\bibitem{WHO} WHO. \emph{Millennium Development Goals (MDGs)}. Geneva, Switzerland: World Health Organization, 2000.
\bibitem{WorldBank} The World Bank. World Development Indicators/\emph{Level \& Trends in Child Mortality}. Report 2011.(accessed May 2012) \verb'http://data.worldbank.org/indicator'
\bibitem{Julie} Julie Knoll Rajaratnam, Jake R Marcus, Abraham D Flaxman, Haidong Wang, Alison Levin-Rector, Laura Dwyer, Megan Costa, Alan D Lopez, Christopher J L Murray. Neonatal, postneonatal, childhood, and under-5 mortality for 187 countries, 1970--2010: a systematic analysis of progress towards Millennium Development Goal 4. \emph{Lancet} 2010; 375: 1988--2008.
\bibitem{SinghGK} Singh GK, van Dyck PC. Infant Mortality in the United States, 1935--2007: Over Seven Decades of Progress and Disparities. A 75th Anniversary Publication. Health Resources and Services Administration, Maternal and Child Health Bureau. Rockville, Maryland: U.S. Department of Health and Human Services; 2010.
\bibitem{Neil} Neil Z Miller, Gary S Goldman. Infant mortality rates regressed against number of vaccine doses routinely given: Is there a biochemical or synergistic toxicity?\emph{ Human and Experimental Toxicology} 2011; 30, 9: 1420--1428.
\bibitem{CIA} Central Intelligence Agency (2011). ``Appendix B. International Organizations and Groups".\emph{The World Factbook}. \verb'https://www.cia.gov/library/publications/the-world-factbook/'
\bibitem{Fox} Fox J. (2002), \emph{Time-series regression and generalized least squares.} Appendix to:  An R and S-plus companion to regression. Sage, Thousand Oaks, CA.
\bibitem{Chris} Chatfield C. (2000), \emph{Time-Series Forecasting}. London:Chapman and Hall/CPC.
\bibitem{Bhargava} Bhargava A., Franzini L., Narendranathan W.(1982), Serial Correlation and the Fixed Effects Model. \emph{Review of Economic Studies}, 49, 533--549.
\bibitem{Peter1} Peter J. Brockwell, Richard A. Davis (2002), \emph{Introduction to Time Series and Forecasting (Second Edition)}. New York: Springer-Verlag.
\bibitem{Peter2} Peter J. Brockwell, Richard A. Davis (1991), \emph{Time Series: Theory and Methods (Second Edition).} Springer.
\bibitem{Diggle} P. J. Diggle (1990), \emph{Time Series: A Biostatistical Introduction}. Oxford University Press
\bibitem{correlogram} Bracewell, R. (1965) \emph{The Autocorrelation Function.} The Fourier Transform and Its Applications. New York: McGraw-Hill, pp. 40-45.
\end{thebibliography}


\newpage
\section*{Figures}

%%% Use this if adding the figures directly in the mansucript, if so, please remember to also upload the files when submitting your article
%%% There is no need for adding the file termination, as long as you indicate where the file is saved. In the examples below the files (logo1.jpg and logo2.eps) are in the Frontiers LaTeX folder
%%% If using *.tif files convert them to .jpg or .png



\begin{figure}[h]
\begin{center}
  \includegraphics[scale=0.5]{1.overall.pdf}
\end{center}
%\caption{Time trends of IMR for different development status}
 \textbf{\refstepcounter{figure}\label{fig:01} Figure \arabic{figure}.}{ Time trends of IMR for different development status }
\end{figure}

%
% For two-column wide figures use
\begin{figure}[H]
  \includegraphics[width=0.5\textwidth]{2.us_ols1.pdf}
  \includegraphics[width=0.5\textwidth]{3.us_ols2.pdf}
\textbf{\refstepcounter{figure}\label{fig:02} Figure \arabic{figure}.}{ Regression line of IMR v.s. time from OLS (left) \& GLS (right) models, United States (1950--2010) }
\end{figure}

\begin{figure}[H]
\begin{center}
  \includegraphics[scale=0.7]{4.afarica.pdf}
\end{center}
\textbf{\refstepcounter{figure}\label{fig:03} Figure \arabic{figure}.}{ IMR time trends for African countries: Algeria, Sudan, Zambia, Morocco, and Libya }       
\end{figure}


\begin{figure}[H]
\begin{center}
  \includegraphics[scale=0.7]{5.asia.pdf}
\end{center}
\textbf{\refstepcounter{figure}\label{fig:04} Figure \arabic{figure}.}{ IMR time trends for Asian countries: China, Japan, India, Singapore, and Thailand }       
\end{figure}


\begin{figure}[H]
\begin{center}
  \includegraphics[scale=0.7]{6.europ.pdf}
\end{center}
\textbf{\refstepcounter{figure}\label{fig:05} Figure \arabic{figure}.}{ IMR time trends for European countries: Poland, Sweden, Italy, France, UK, and Russia }       
\end{figure}

\begin{figure}[H]
\begin{center}
  \includegraphics[scale=0.7]{7.america.pdf}
\end{center}
\textbf{\refstepcounter{figure}\label{fig:06} Figure \arabic{figure}.}{ IMR time trends for American and Oceanian countries: US, Brazil, Mexico, and Australia }       
\end{figure}




%\begin{figure}
%\begin{center}
%\includegraphics[width=3.5cm]{logo2}% This is an *.eps file
%\end{center}
% \textbf{\refstepcounter{figure}\label{fig:02} Figure \arabic{figure}.}{ Enter the caption for your figure here.  Repeat as  necessary for each of your figures }
%\end{figure}

%%% If you don't add the figures in the LaTeX files, please upload them when submitting the article.

%%% Frontiers will add the figures at the end of the provisional pdf automatically %%%

%%% The use of LaTeX coding to draw Diagrams/Figures/Structures should be avoided. They should be external callouts including graphics.

\end{document}
